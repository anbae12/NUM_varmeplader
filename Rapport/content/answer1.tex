\section*{Radiation Balance Between Parallel Plates}
Assignment is completed by Mikkel Jaedicke (mijae12) \& Anders Bæk (anbae12)

The radiation balance is covered by equations \ref{eq:1}-\ref{eq:3}. 

\begin{equation}
\begin{align*}
u\left( x \right) &=\varepsilon _{ 1 }\sigma { T }_{ 1 }^{ 4 }+\left( 1-\varepsilon _{ 1 } \right) \int _{ -\frac { 1 }{ 2 } w }^{ \frac { 1 }{ 2 } w } F\left( x,y,d \right) \cdot v\left( y \right) { dy }\\ 
v\left( y \right) &=\varepsilon _{ 2 }\sigma { T }_{ 2 }^{ 4 }+\left( 1-\varepsilon _{ 2 } \right) \int _{ -\frac { 1 }{ 2 } w }^{ \frac { 1 }{ 2 } w } F\left( x,y,d \right) \cdot u\left( x \right) { dx }
\end{align*}
\label{eq:1}
\end{equation}


\begin{equation}
F\left( x,y,d \right) =  \frac {1}{2}\frac { d^2 }{ \left( d^2+\left( x-y \right)^2 \right)^{\frac{3}{2}}} 
\label{eq:2}
\end{equation}
\begin{equation}
\begin{align*}
I_{ 1 }\left( x \right) &=\int _{ -\frac { 1 }{ 2 } w }^{ \frac { 1 }{ 2 } w } F\left( x,y,d \right)\cdot v\left( y \right) { dy }
\\
I_{ 2 }\left( y \right) &=\int _{ -\frac { 1 }{ 2 } w }^{ \frac { 1 }{ 2 } w } F\left( x,y,d \right)\cdot v\left( x \right) { dx }
\end{align*}
\label{eq:3}
\end{equation}
\textbf{DATA:} \( T_1 = 1000\), \(  T_2=500 \), \(  \varepsilon_1 = 0.80 \), \(  \varepsilon_2 = 0.60 \), \(  \sigma = 1.7212\cdot 10^{-9} \), \( d=1.0  \), \(  w=1.0 \)
\section*{Question 1}

Equation \ref{eq:1} is rearranged to give equation \ref{eq:4}.

\begin{equation}
\begin{align*}
u\left( x \right)- \left( 1-\varepsilon _{ 1 } \right) \int _{ -\frac { 1 }{ 2 } w }^{ \frac { 1 }{ 2 } w } F\left( x,y,d \right) \cdot v\left( y \right) { dy } &=\varepsilon _{ 1 }\sigma { T }_{ 1 }^{ 4 }\\ 
v\left( y \right) -\left( 1-\varepsilon _{ 2 } \right) \int _{ -\frac { 1 }{ 2 } w }^{ \frac { 1 }{ 2 } w } F\left( x,y,d \right) \cdot u\left( x \right) { dx } &=\varepsilon _{ 2 }\sigma { T }_{ 2 }^{ 4 }
\end{align*}
\label{eq:4}
\end{equation}

The integral needs to be discretized in order to make it into a system of linear equations (SLE). The trapez-method is used to obtain equation \ref{eq:5}.

\begin{equation}
\begin{align*}
u\left( x \right)- \left( 1-\varepsilon _{ 1 } \right) + h \cdot ( \frac{1}{2} \cdot F(x,y_0,d) \cdot v(y_0) + \frac{1}{2} \cdot F(x,y_N,d) \cdot v(y_N) +  \sum\limits_{i=1}^{N-1} F(x,y_i,d) \cdot v(y_i) ) &=\varepsilon _{ 1 }\sigma { T }_{ 1 }^{ 4 }\\ 
v\left( x \right)- \left( 1-\varepsilon _{ 2 } \right) + h \cdot ( \frac{1}{2} \cdot F(x_0,y,d) \cdot u(x_0) + \frac{1}{2} \cdot F(x_N,y,d) \cdot u(x_N) +  \sum\limits_{i=1}^{N-1} F(x_i,y,d) \cdot u(x_i) ) &=\varepsilon _{ 2 }\sigma { T }_{ 2 }^{ 4 }
\end{align*}
\label{eq:5}
\end{equation}

From equation \ref{eq:5} a SLE in the form $A \cdot z = b$ can be made.
\\Equations \ref{eq:A}-\ref{eq:b} illustrates a SLE for \( N = 4 \).
The C++ implementation of equation \ref{eq:A} and equation \ref{eq:b} is scaleable (see enclosed main.cpp, line 39-84, line 93-104). \\
Equation \ref{eq:x} display the returned \texttt{VecDoub} by SVD solve function. 

\begin{equation}
 A = \begin{bmatrix}
   0  &   1  &  0  &  0  &  0  & - \frac{1}{2} \cdot \beta_1 \cdot f(x_1,y_0, d)  &   - \beta_1 \cdot f(x_1,y_1, d)  &   - \beta_1 \cdot f(x_1,y_2, d)  &   - \beta_1 \cdot f(x_1,y_3, d)  &   - \frac{1}{2} \cdot \beta_1 \cdot f(x_1,y_4, d)\\
   0  &   0  &  1  &  0  &  0  & - \frac{1}{2} \cdot \beta_1 \cdot f(x_2,y_0, d)  &   - \beta_1 \cdot f(x_2,y_1, d)  &   - \beta_1 \cdot f(x_2,y_2, d)  &   - \beta_1 \cdot f(x_2,y_3, d)  &   - \frac{1}{2} \cdot \beta_1 \cdot f(x_2,y_4, d)\\
   0  &   0  &  0  &  1  &  0  & - \frac{1}{2} \cdot \beta_1 \cdot f(x_3,y_0, d)  &   - \beta_1 \cdot f(x_3,y_1, d)  &   - \beta_1 \cdot f(x_3,y_2, d)  &   - \beta_1 \cdot f(x_3,y_3, d)  &   - \frac{1}{2} \cdot \beta_1 \cdot f(x_3,y_4, d)\\
   0  &   0  &  0  &  0  &  1  & - \frac{1}{2} \cdot \beta_1 \cdot f(x_4,y_0, d)  &   - \beta_1 \cdot f(x_4,y_1, d)  &   - \beta_1 \cdot f(x_4,y_2, d)  &   - \beta_1 \cdot f(x_4,y_3, d)  &   - \frac{1}{2} \cdot \beta_1 \cdot f(x_4,y_4, d)\\
   - \frac{1}{2} \cdot \beta_2 \cdot f(x_0,y0, d)      &   - \beta_2 \cdot f(x_1,y_0, d)  &   - \beta_2 \cdot f(x_2,y_0, d)  &   - \beta_2 \cdot f(x_4,y_0, d)  &  - \frac{1}{2} \cdot \beta_2 \cdot f(x_4,y_0, d) &  1  &   0  &  0  &  0  &  0\\
   - \frac{1}{2} \cdot \beta_2 \cdot f(x_0,y_1, d)    &   - \beta_2 \cdot f(x_1,y_1, d)  &   - \beta_2 \cdot f(x_2,y_1, d)  &   - \beta_2 \cdot f(x_4,y_1, d)  &   - \frac{1}{2} \cdot \beta_2 \cdot f(x_4,y_1, d) &  0  &   1  &  0  &  0  &  0\\
   - \frac{1}{2} \cdot \beta_2 \cdot f(x_0,y_2, d)    &   - \beta_2 \cdot f(x_1,y_2, d)  &   - \beta_2 \cdot f(x_2,y_2, d)  &   - \beta_2 \cdot f(x_4,y_2, d)  &   - \frac{1}{2} \cdot \beta_2 \cdot f(x_4,y_2, d) &  0  &   0  &  1  &  0  &  0\\
   - \frac{1}{2} \cdot \beta_2 \cdot f(x_0,y_3, d)    &   - \beta_2 \cdot f(x_1,y_3, d)  &   - \beta_2 \cdot f(x_2,y_3, d)  &   - \beta_2 \cdot f(x_4,y_3, d)  &   - \frac{1}{2} \cdot \beta_2 \cdot f(x_4,y_3, d) &  0  &   0  &  0  &  1  &  0\\
   - \frac{1}{2} \cdot \beta_2 \cdot f(x_0,y0_4, d)  &   - \beta_2 \cdot f(x_1,y_4, d)  &   - \beta_2 \cdot f(x_2,y_4, d)  &   - \beta_2 \cdot f(x_4,y_4, d)  &   - \frac{1}{2} \cdot \beta_2 \cdot f(x_4,y_4, d) &  0  &   0  &  0  &  0  &  1 \\
\end{bmatrix}
\label{eq:A}
\end{equation}

\textbf{where:} \( \beta_1 = (1-\varepsilon_1) \cdot h \),   \(\beta_2 = (1-\varepsilon_2) \cdot h\) and   \( h=\frac { \left( \frac { 1 }{ 2 } w-\left( -\frac{1}{2} w \right)  \right)  }{ N }  \)

\begin{equation}
 z ={ \begin{bmatrix}
 u_0 & 
u_1 &
 u_2 &
 u_3 &
 u_4 &
 v_0 &
 v_1 &
 v_2 &
 v_3 &
 v_4 
 \end{bmatrix} }^{ T } 
 \label{eq:x}
\end{equation}

\begin{equation}
 b ={ \begin{bmatrix}
 \varepsilon_1 \sigma T_1^4 &
 \varepsilon_1 \sigma T_1^4&
 \varepsilon_1 \sigma T_1^4&
 \varepsilon_1 \sigma T_1^4&
 \varepsilon_1 \sigma T_1^4&
 \varepsilon_2 \sigma T_2^4&
 \varepsilon_2 \sigma T_2^4&
 \varepsilon_2 \sigma T_2^4&
 \varepsilon_2 \sigma T_2^4&
 \varepsilon_2 \sigma T_2^4
 \end{bmatrix} }^{ T } 
 \label{eq:b}
\end{equation}


\newpage
\section*{Question 2}
SVD is used to solve the SLE from question 1. Results are shown in table \ref{tb:resultater}.
$Q_1$ and $Q_2$ needs to be calculated. A rearrangement of the equation for $Q_1$ is made in equation \ref{eq:ss}.	A similar rearrangement is made for the equation for $Q_2$.

The constants \( c_{11} \) and \( c_{12} \) belongs to the equation for \( Q_1 \) and \( c_{21} \) and \( c_{22} \) belongs to the equation for \( Q_2 \).

\begin{equation}
\begin{align*}
Q_{ 1 }&=\int _{ -\frac { 1 }{ 2 } w }^{ \frac { 1 }{ 2 } w } \left( { u }\left( x \right) -I\left( x \right)  \right) { dx }\\ 
u\left( x \right) &={ c }_{ 11 }+{ c }_{ 12 }\cdot I_{ 1 }\\ 
{ I }_{ 1 }&=\frac { u\left( x \right) -{ c }_{ 11 } }{ { c }_{ 12 } } \\ 
Q_{ 1 }&=\int _{ -\frac { 1 }{ 2 } w }^{ \frac { 1 }{ 2 } w } \left( { u }\left( x \right) -\frac { u\left( x \right) -{ c }_{ 11 } }{ { c }_{ 12 } }  \right) { dx }
\end{align*}
\label{eq:ss}
\end{equation}

\begin{equation}
\begin{align*}
{ c }_{ 11 }&={ \varepsilon  }_{ 1 }\cdot \sigma \cdot { T }_{ 1 }^{ 4 }\\ 
{ c }_{ 12 }&=\left( 1-{ \varepsilon  }_{ 1 } \right) \\ 
{ c }_{ 21 }&={ \varepsilon  }_{ 2 }\cdot \sigma \cdot { T }_{ 2 }^{ 4 }\\ 
{ c }_{ 22 }&=\left( 1-{ \varepsilon  }_{ 2 } \right) 
\end{align*}
\label{eq:ad}
\end{equation}

The integrals in equation \ref{eq:ss} need to be discretized. This is done with the trapez-method and the resulting equations can be seen in equation \ref{eq:Q12}. The results for $Q_1$ and $Q_2$ can be seen in table \ref{tb:resultater1}.

\begin{equation}
\begin{align*}
{ Q }_{ 1 }&=\frac { 1 }{ 2 } \cdot h\cdot \left( u\left( { x }_{ 0 } \right) -\left( \frac { u\left( { x }_{ 0 } \right) -{ c }_{ 11 } }{ { c }_{ 12 } }  \right) +u\left( { x }_{ N } \right) -\left( \frac { u\left( { x }_{ N } \right) -{ c }_{ 11 } }{ { c }_{ 12 } }  \right)  \right) +h\sum _{ i=1 }^{ N-1 }{ u\left( { x }_{ i } \right) - } \left( \frac { u\left( { x }_{ i } \right) -{ c }_{ 11 } }{ { c }_{ 12 } }  \right) \\
{ Q }_{ 2 }&=\frac { 1 }{ 2 } \cdot h\cdot \left( v\left( { y }_{ 0 } \right) -\left( \frac { v\left( { y }_{ 0 } \right) -{ c }_{ 21 } }{ { c }_{ 22 } }  \right) +v\left( { y }_{ N } \right) -\left( \frac { v\left( { y }_{ N } \right) -{ c }_{ 21 } }{ { c }_{ 22 } }  \right)  \right) +h\sum _{ i=1 }^{ N-1 }{ v\left( { y }_{ i } \right) - } \left( \frac { v\left( { y }_{ i } \right) -{ c }_{ 21 } }{ { c }_{ 22 } }  \right)  
\end{align*}
\label{eq:Q12}
\end{equation}

The values for \( u\left( x \right)  \) and  \( v\left( y \right)  \), \( x=y= \pm0.5  \), \( x=y= \pm0.25  \) and \( x=y= 0  \)  for a given \( N \) is showed in table \ref{tb:resultater}.

\begin{table}[th!]
\centering
\begin{tabular}{c|c|c|c|c|c|c|c|c|c|c}
 N &  \( u(-0.50) \) & \( u(-0.25) \) & \( u(0.0) \) & \( u(0.25) \) & \( u(0.50) \) & \( v(-0.50) \) & \( v(-0.25) \) & \( v(0.0) \) & \( v(0.25) \) & \( v(0.50) \)  \\\hline
4&1.39015295e+03&1.39407857e+03&1.39407857e+03&1.39407857e+03&1.39015295e+03&2.60504838e+02&2.97020995e+02&3.11005259e+02&2.97020995e+02&2.60504838e+02\\
8&1.39040598e+03&1.39560589e+03&1.39444651e+03&1.39444651e+03&1.39040598e+03&2.61150040e+02&2.82020745e+02&3.12971204e+02&3.09290227e+02&2.61150040e+02
\\
16&1.39046959e+03&1.39600813e+03&1.39453871e+03&1.39453871e+03&1.39046959e+03&2.61311714e+02&2.72227782e+02&3.13460057e+02&3.12528101e+02&2.61311714e+02\\
32&1.39048552e+03&1.39610999e+03&1.39456177e+03&1.39456177e+03&1.39048552e+03&2.61352155e+02&2.66906623e+02&3.13582105e+02&3.13348382e+02&2.61352155e+02\\
64&1.39048950e+03&1.39613554e+03&1.39456754e+03&1.39456754e+03&1.39048950e+03&2.61362267e+02&2.64160578e+02&3.13612607e+02&3.13554131e+02&2.61362267e+02
 \\
128&1.39049050e+03&1.39614193e+03&1.39456898e+03&1.39456898e+03&1.39049050e+03&2.61364795e+02&2.62768839e+02&3.13620232e+02&3.13605610e+02&2.61364795e+02 \\ 
256&1.39049075e+03&1.39614353e+03&1.39456934e+03&1.39456934e+03&1.39049075e+03&2.61365427e+02&2.62068623e+02&3.13622138e+02&3.13618483e+02&2.61365427e+02\\
512&1.39049081e+03&1.39614393e+03&1.39456943e+03&1.39456943e+03&1.39049081e+03&2.61365585e+02&2.61717471e+02&3.13622615e+02&3.13621701e+02&2.61365585e+02 \\
1024&1.39049083e+03&1.39614403e+03&1.39456945e+03&1.39456945e+03&1.39049083e+03&2.61365624e+02&2.61541638e+02&3.13622734e+02&3.13622505e+02&2.61365624e+02 \\
\end{tabular}
\caption[tekst i indholdsfortegnelsen]{Values for \( u\left( x \right)  \) and  \( v\left( y \right)  \), \( x=y= \pm0.5  \), \( x=y= \pm0.25  \) and \( x=y= 0  \)  for a given \( N \).}
\label{tb:resultater}
\end{table}

\newpage
\section*{Question 3}
The Richardson $\alpha^k$ estimate and Richardson error estimate is shown in \ref{eq:EE}. The $\alpha^k$ estimate is seen in  table \ref{tb:resultater1} to converge to 4. Thus $\alpha^k = 4$ is used in the Richarson error estimate.

\begin{equation}
\begin{align*}
\alpha ^{ k }&=\frac { S\left( h_{ 1 } \right) -S\left( h_{ 2 } \right)  }{ S\left( h_{ 2 } \right) -S\left( h_{ 3 } \right)  } \\
EE&=\frac { S\left( h_{ 2 } \right) -S\left( h_{ 1 } \right)  }{ \alpha ^{ k }-1 } 
\end{align*}
\label{eq:EE}
\end{equation}

\( Q_1  \), \( Q_2  \), \( EE \) and \( \alpha^k \) is calculated for a given \( N \), and the results can be seen in table \ref{tb:resultater1}. It is clearly seen that the estimated error decreases when $N$ is increased.


\begin{table}[th!]
\centering
\begin{tabular}{c|c|c|c|c|c|c}
 N &  \( Q_1 \) & \( { \alpha }_{  1}^{ k } \) & \(  EE_1 \) & \( Q_2 \) & \(  { \alpha }_{  2}^{ k } \) & \( EE_2 \)  \\\hline
4&1.27409460e+03&-0.00000000e+00&4.24698201e+02&-2.76582033e+02&0.00000000e+00&-9.21940109e+01\\
8&1.27208978e+03&-6.35513621e+02&-6.68275530e-01&-2.80870375e+02&6.44962508e+01&-1.42944760e+00\\
16&1.27158320e+03&3.95762143e+00&-1.68857871e-01&-2.81952494e+02&3.96291264e+00&-3.60706311e-01\\
32&1.27145623e+03&3.98967651e+00&-4.23237000e-02&-2.82223632e+02&3.99103702e+00&-9.03790941e-02\\
64&1.27142447e+03&3.99743475e+00&-1.05877150e-02&-2.82291454e+02&3.99777714e+00&-2.26073368e-02\\
128&1.27141653e+03&3.99935965e+00&-2.64735257e-03&-2.82308412e+02&3.99944538e+00&-5.65261796e-03\\
256&1.27141454e+03&3.99983995e+00&-6.61864625e-04&-2.82312651e+02&3.99986141e+00&-1.41320345e-03 \\
512&1.27141405e+03&3.99996020e+00&-1.65467803e-04&-2.82313711e+02&3.99996540e+00&-3.53303919e-04 \\
1024&1.27141392e+03&3.99999013e+00&-4.13670527e-05&-2.82313976e+02&3.99999125e+00&-8.83261731e-05 \\
\end{tabular}
\caption[tekst i indholdsfortegnelsen]{\( Q_1  \), \( Q_2  \), \( EE \) and \( \alpha^k \) is calculated for a given \( N \).}
\label{tb:resultater1}
\end{table}

 The code\footnote{The complete code implementation can by found inside the zipped file.} for solving the SLE, calculating $Q_1$ and $Q_2$ and calculating the Richardson $\alpha^k$ and error estimate is shown in code snippet \ref{code:UNOtimerTestpy}.

\newpage
%%%% USE THIS FOR INSERT CODE!!
\lstinputlisting[firstline=139,firstnumber=139,lastline=173
,caption={Code snippet of main.cpp, line 139-173.},label=code:UNOtimerTestpy]
{../../mijae12_anbae12/Varmeplader/Varmeplader/main.cpp}


